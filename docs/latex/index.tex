\subsubsection*{A dynamic model of the replication process in kinetoplastida}

Re\+Dy\+Mo is a Python-\/coded, stochastic dynamic model simulator that reproduces the D\+NA replication process of cellular organisms belonging to the kinetoplastida group. Initially, we focused on {\itshape Trypanosoma brucei} strain {\itshape T\+R\+E\+U927}.

This simulator is capable of\+:
\begin{DoxyItemize}
\item Calculation of various relevant properties about the replication process, including S-\/phase duration, mean inter-\/origin distance between replication origin sites, the achieved replication percentage given a time limit, etc.
\item Storage of information about each step of the simulation, such as the replication stage at each model simulation iteration and whether any collisions happened during a given iteration.
\item Storage of information regarding the replication timing program, that is, for each base pair it is recorded at which simulation iteration it was replicated.
\item Simulation of replication-\/transcription conflicts with a variety of input parameters like replication machinery speed and transcription frequency.
\item Simulation of dormant origins firing.
\end{DoxyItemize}

\subsubsection*{Requirements}

To run Re\+Dy\+Mo, you only need a system with Python3 installed, which is done by default in most Linux distributions.

\subsubsection*{Database}

The system uses a simple {\itshape S\+Q\+Lite} database. Python already has plenty of functionalities to access and modify {\itshape S\+Q\+Lite} databases. However, if it is necessary to visualize the data in a more intuitive fashion, we recommend the usage of a third-\/party software such as \href{https://sqlitestudio.pl/index.rvt>}{\tt S\+Q\+Lite\+Studio}.

\subsubsection*{Parameters}

In this version of Re\+Dy\+Mo, all parameters are mandatory and are listed below\+:
\begin{DoxyItemize}
\item {\itshape organism}\+: Organism name, as saved in the database (remember to add single quotation marks when using space-\/separated names).
\item {\itshape cells}\+: Number of independent simulations to be made.
\item {\itshape resources}\+: Number of available forks for the replication process.
\item {\itshape speed}\+: Movement speed of each replication machinery (in bases per second).
\item {\itshape period}\+: Time between consecutive activations of a transcription region (in seconds).
\item {\itshape timeout}\+: Maximum allowed number of iterations of a simulation; if this value is reached, then a simulation is ended even if D\+NA replication is not completed yet.
\item {\itshape dormant}\+: Flag that either activates (\textquotesingle{}True\textquotesingle{}) or disables (\textquotesingle{}False\textquotesingle{}) the firing of dormant origins.
\end{DoxyItemize}

\subsubsection*{Running the simulation}

To run the program, the syntax of the main simulator program is the following one\+: 
\begin{DoxyCode}
$ ./main.py --organism 'organism' --resources resources\_value --speed speed\_value --cells numbe\_of\_cells
       --period period\_value --timeout timeout\_value --dormant [True|False]
\end{DoxyCode}


The command above must be executed within the \char`\"{}src\char`\"{} directory. For example, to run a simulation of seven cells of {\itshape T. brucei T\+R\+E\+U927}, with 10 forks, replisome speed of 65 bp/sec, transcription frequency of 150 sec, a timeout of one million iterations and no dormant origin firing, one must type\+: 
\begin{DoxyCode}
$ cd src
$ ./main.py --organism 'Trypanosoma brucei brucei TREU927' --resources 10 --speed 65 --period 150 --cells 7
       --timeout 1000000 --dormant False
\end{DoxyCode}
 The simulation results will be stored into a directory named {\itshape output/\+False\+\_\+10\+\_\+50/}, in which \char`\"{}output\char`\"{} is the outer directory name and the inner directory name of composed of the concatenation of the used parameter values for dormant origin firing, resources and period.

\subsubsection*{Aggregating the simulation results}

If more than one cell is simulated at once, then the results may be averaged through the usage of an aggregator script, whose syntax is the following\+: 
\begin{DoxyCode}
$ cd script
$ ./cell\_output\_aggregator.py *output\_directory* > *aggregation\_file\_and\_path*
\end{DoxyCode}
 where {\itshape output\+\_\+directory} is the output directory of the simulation and {\itshape aggregation\+\_\+file\+\_\+and\+\_\+path} is both the path and name for the file containing the result of data aggregation. For example, to aggregate the aforementioned example, one could just type\+: 
\begin{DoxyCode}
$ cd script
$ ./cell\_output\_aggregator.py ../output/False\_10\_150 > ../output/aggregated.txt
\end{DoxyCode}


\subsubsection*{Bug report and contact}

If you have any bug report and/or want to contact for other subjects (e.\+g., to collaborate in this project), please do not hesitate to contact us!

Please, address your message to\+:

msreis at butantan dot gov dot br. 